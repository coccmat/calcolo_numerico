\documentclass[../main.tex]{article}
\usepackage[utf8]{inputenc}
\usepackage{hyperref}
\usepackage{mdframed}
\usepackage{amsmath}

\title{Sistemi lineari}
\begin{document}
\author{Matteo Cocciniglia}
\date{02/03/2022}
\maketitle
\section{matrici simmetriche}
look at pag 167\\%DELL
Ricordando che:\begin{itemize}
\item Una matrice quadrata $A$ è simmetrica se $A=A^T$.
\item Per memorizzare una matrice in modo ottimizzato è richiesto $\simeq \frac{n^2}{2}$ locazioni di memoria.
\end{itemize} 
\subsection{Teorema di fattorizzazzione di Gauss per matrici simmetriche}
\begin{mdframed}

Se $A$ è simmetrica e tutti i suoi  minori principali sono diversi da zero, allora\\
esistono una matrice triangolarte inferiore $L$ una diagonale $D$ con elementi $\neq 0$\\
tali che $A=LDL^T$
\end{mdframed}
\subsection{dimsotrazione}
Sfruttando l'ipotesi che tutti i minori di $A$ sono $\neq 0$  dai risultati del \textbf{teorema di fattorizazzione di Gauss} %%da scrivere
si ottiene che $A=LU$ dove $L$ è una matrice triangolare inferiore con diagonale unitaria e $U$ matrice diagonale superiore.\\
Si definisce  $D=\begin{pmatrix}
  u_{11} & 0  & \cdots & \\
  0 & u_{22}  & & \\
  \vdots & & \ddots & \\
  &  &   & u_{nn}\\
\end{pmatrix}$,quindi si ha che $A = LDD^{-1}U$\\ 
dato che $DD^{-1}$ è la matrice identità il risultato non è alterato.\\
Provare che $D^{-1}U\stackrel{?}{=}L^{T}$:\\ Sfruttando la simmetria della matrice $A=A^T$ si ottiene $A=(LDD^{-1}U)^T$. Quindi seguendo i passi algebrici $L^{T}D(D^{-1}U)^T \rightarrow \texbf{LL^{-1}}LDD^{-1}U = L^{-1}L^{T}D(D^{-1}U)^T$, ricordando che $LL^{-1}$ è la matrice Identità si arriva alla seguente uaglianza $D(D^{-1}U)^T=L^TL^{-1}LDD^{-1}U$ inoltre semplificando $L^T$ moltiplicandolo per la sua inversa si arriva al risultato finale $D(D^{-1}U)^TL^{T^{-1}}=L^{-1}DD^{-1}U$, notando che  $D(D^{-1}U)^TL^{T^{-1}} $ è una matrice triangolare superiore e che $L^{-1}DD^{-1}U$ è una triangolare inferiore si ha che tutti gli elementi non nella diagonale sono uguali a zero; quindi $D(D^{-1}U)^TL^{T^{-1}}$ è una matrice diagonale da qui otteniamo che $(D^{-1}U)^TL^{T^{-1}}$ è anchessa una diagonale e sicuramente anche unitaria dato il risultato appena ottenuto. Quindi possiamo concludere che $D(D^{-1}U)^TL^{T^{-1}} = I \rightarrow D(D^{-1}U)^T = L^{T^{-1}}=L^{T}$   

\end{document}
